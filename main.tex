% vim: spell spelllang=en:
\input{preamble}

\title{
    GCS - Quantum Computing
}
\author {
    Aleix Boné\\
    Javier Cabrera\\
    Victor Guardia\\
    Albert Mercadé
}
\date{
    \today
}

\begin{document}

% vim: spell spelllang=en:
%! TEX root = **/main.tex

% Cover with title of work, name of course, data and list of working
% team members by alphabetical order of family name

\thispagestyle{empty}
\clearpage
\setcounter{page}{-1}

\begin{titlepage}
{
    \centering
    \null
    \vfill
    {\Large GCS\par}
    \vspace{2em}
    {\Huge \bfseries
        Quantum Computing
    \par}
    \vspace{2em}
    {\large \scshape
        \today
    \par}
    \vfill
\begin{center}

\end{center}
    \vspace{3cm}

    \vfill
    {\raggedleft \large
Aleix Boné\\
Javier Cabrera\\
Victor Guardia\\
Albert Mercadé
        \par}
}
\end{titlepage}


\section{Introduction}%
\label{sec:introduction}

\section{Elliptic Curve Cryptography (ECC)}

ECC is a form of public key cryptography (or asymmetric) based on the complexity
of computing discrete logarithms in elliptic curve groups. (Elliptic Curve
Discrete Logarithm Problem).

Although elliptic curves have been studied since the work of Diophantus in the
second century A.D., its use in cryptography was not discovered until 1984.
Victor Miller and Neal Koblitz proposed two different methods to apply elliptic
curves to cryptography.

The main attractiveness of ECC is that unlike with RSA, there are no known
sub-exponential time algorithms to solve ECDLP for properly chosen elliptic
curves. This makes it possible to have similar security with smaller parameters.
Moreover the computations needed to encrypt and decrypt are faster.

For instance, if we consider a standard 256bit ECDSA, the computing power needed
to crack with known algorithms is equivalent to a 3072 bit RSA.

Nowadays, ECC is used all around the internet from TLS, SSH, Bitcoin, national
ID cards, the Tor anonymity network, and WhatsApp among others. Along with RSA,
ECC has become one of the cornerstones of the internet as we know it today.

\subsection{Quantum}

Despite the fact that with the know traditional algorithms, ECC is far superior
to RSA in terms of security, against the threat of quantum computing it is
equally vulnerable.

Shor published 2 algorithms in his famous paper: "Polynomial-Time Algorithms for
Prime Factorization and Discrete Logarithms on a Quantum Computer". Which can be
used to crack RSA and ECC in polynomial time. However, the cost of those
algorithms differ vastly.

As shown by Roetteler et al. (2017), ECC 256 requires $10^{11}$  toffoli gates
whereas RSA 3072 requires $10^{13}$. RSA provides greater security than ECC
using Shor's algorithms. However, there are improvements and optimizations that
can be applied to Shor's original prime factorization algorithm which reduce the
order of RSA by 100, making it comparable to ECC.

With current technology we are still very far away from being able to build a
quantum computer with sufficient qubits to crack either RSA or ECC, but progress
is being made rapidly, not only hardware wise but also on algorithms and
techniques (we went from needing 1 billion qubits to 20 million in the space of 7 years).

Some experts theorize that quantum cracking of RSA and ECC may be a real threat
as early as 2030.


\section{Multivariate Public Key Cryptography (MVPKC)}

Multi variate public key cryptography is a method of public key cryptography
based on the NP-hardness of solving nonlinear equations over a finite field.
The fundamental principle of the method is rooted on algebraic geometry, a field
of mathematics that developed in the 20th century (and not numerical algebra or
eliptic curves). There are no known quantum algorithms that can break MVPKC
which makes it one of the candidates for post quantum cryptography methods.

We will not enter into details of how MVPKC is implemented.  There are various
forms of MVPKC which differ on how the trapdoor one way function is defined.

One of the main benefits of MVPKC over lattice and other post quantum
cryptography methods is that it takes less computing resources although the keys
are considerably larger (reaching Kbs in size). This makes it unsuitable for
really small integrated devices with low memory.



\end{document}
